\documentclass{article}
\usepackage[german]{babel}
\author{Phillip Eckstein}
\title{Exposé zum Thema Java Profiling}

\addto\captionsgerman{%
  \renewcommand{\contentsname}%
    {Inhalt}%
}

\begin{document}
    \selectlanguage{german}
    \maketitle
    \tableofcontents

    \pagebreak

    \section{Allgemeine Aufgabenstellungen}
    \subsection{Zielstellung}
    Ziel ist es herrauszufinden, welche Methoden und Klassen des ZUUL-Projekst aus Modul Objektorientierte Softwareentwicklung die meisten Ressourcen verbrauchen.

    \subsection{Aufgaben}
    Um dieses Vorhaben umzusetzen muss ein geeigneter Profiler ausgewählt werden. Bei der Auswahl sind bestimmte Kriterien und Anforderungen an den Profiler zu beachten. Diese Anforderungen sind zu Formulieren. Die Auswahl ist zu begründen, gegebenenfalls soll ein Vergleich zwischen verschiedenen Optionen erstellt werden. Im Vorfeld muss geklärt werden, was genau Profiling ist. Die durchführung des Profilings ist zu beschreiben. Das Profiling soll an meheren ständen des projekts durchgeführt werden, um eventuelle veränderungen aufzuzeigen.

    \section{Gliederung}
    \begin{itemize}
      \item Einleitung
      \item Begriffserklärung
      \item Anforderungsanalyse
      \item Profiler Auswahl
      \item durchführung
      \item Auswertung und Fazit
    \end{itemize}

    \section{Vorgehen}
    Als erstes müssen die Begrifflichkeiten geklärt werden. Danach sind die Gegebenheiten des ZUUL Projekts zu analysieren und darauf bauend die Anforderungen an den Profiler zu formulieren. Der nächste schritt besteht darin, basierend auf den Anfordeungen einen Profiler auszuwählen. Danach sind mehrere Profiling-Durchläufe in den verschiedenen Phasen der Entwicklung des ZUUL Projekts duchzuführen und aufzuzeichnen. Als letzer Schritt sind die Ergebnisse Auszuwerten.
    
\end{document}