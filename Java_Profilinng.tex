\documentclass{article}
\usepackage[german]{babel}
\author{Phillip Eckstein}
\title{Java Profiling}

\addto\captionsgerman{%
  \renewcommand{\contentsname}%
    {Inhalt}%
}


\begin{document}
    
\selectlanguage{german}
\maketitle
\tableofcontents

\pagebreak

\section{Einleitung}
Diese Arbeit beschaeftigt sich mit dem Thema des Profiling in Java. Es wird am Beispiel des Zuul Projektes der gesammte vorgang des Profilings beschrieben. Dabei wird auf alle Wichtigen Aspekte der Auswahl und der Durchfuehrung eingegangen. Ausserdem wird anschliessend eine auswertung der ergebisse durchgefuehrt um zu demontrieren was man duchr Profiling herrausfinden kann. 

\section{Begriffserklaerung}
IDE: Integraded Development Enviroment

\section{Anforderungsanalyse}
Ein Profiler muss Anforderungen erfuellen. Diese ergeben sich abhaenning von verschiedenen Einfluessen. Dazu zaehlen sowohl auessere Einfluesse als auch innere. Als auessere einfluesse seien Einfuesse anzunehmen, die nicht geaendert werden koennen. Ein Beispiel dafuer sind Unternehmensrichtlienien. Als interne einfluesse sind kriterien wie die Art, Programmiersprache und  Architektur der zu profilenden Software sowie die benutzte IDE. Des weiteren sollte schon im vorraus geklaert werden welche Teile der Anwendung genauer betrachtet werden sollen. Dies ist wichtig, da verschiende Profiler wie nachfolgend beschrieben bestimmte Aspeckte des Programms besser Testen koennen. 

Im Rahmen dieser Arbeit Soll eine Java Anwendung einem Profiling unterzogen werden. Als IDE kommt Intelij Ultimate von JetBrains zum Einsatz. Daraus ergibt sich bereits die erste anforderung: Es wird zwingend ein Profiler benoetigt, der Java versteht. Eine weitere Andorderung koennte sein, das das tool eine moeglichst einfache integration und bedienung bietet um unerfahreneren Anwendern das Arbeiten zu erelichtern 
\section{Profiler-Auswahl}

\section{Durchfuehrung}

\section{Auswertung und Fazit}

\end{document}